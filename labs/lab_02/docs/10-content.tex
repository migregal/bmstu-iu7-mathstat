\chapter*{Задание}

\textbf{Цель работы}: Построение доверительных интервалов для математического ожидания и дисперсии нормальной случайной величины.

\begin{enumerate}
	\item Для выборки объема $n$ из нормальной генеральной совокупности $X$ реализовать в виде программы на ЭВМ
	\begin{enumerate}
		\item вычисление точечных оценок $\hat\mu(\vec X_n)$ и $S^2(\vec X_n)$ математического ожидания $MX$ и дисперсии $DX$ соответственно;
		\item вычисление нижней и верхней границ $\underline\mu(\vec X_n)$, $\overline\mu(\vec X_n)$ для $\gamma$-доверительного интервала для математического ожидания $MX$;
		\item вычисление нижней и верхней границ $\underline\sigma^2(\vec X_n)$, $\overline\sigma^2(\vec X_n)$ для $\gamma$-доверительного интервала для дисперсии $DX$;
	\end{enumerate}
	\item вычислить $\hat\mu$ и $S^2$ для выборки из индивидуального варианта;
	\item для заданного пользователем уровня доверия $\gamma$ и $N$ – объёма выборки из индивидуального варианта:
	\begin{enumerate}
		\item на координатной плоскости $Oyn$ построить прямую $y = \hat\mu(\vec{x_N})$, также графики функций $y = \hat\mu(\vec x_n)$, $y = \underline\mu(\vec x_n)$ и $y = \overline\mu(\vec x_n)$ как функций объема $n$ выборки, где $n$ изменяется от 1 до $N$;
		\item на другой координатной плоскости $Ozn$ построить прямую $z = S^2(\vec{x_N})$, также графики функций $z = S^2(\vec x_n)$, $z = \underline\sigma^2(\vec x_n)$ и $z = \overline\sigma^2(\vec x_n)$ как функций объема $n$ выборки, где $n$ изменяется от 1 до $N$.
	\end{enumerate}
\end{enumerate}

\chapter{Теоретические сведения}

\section{Определение $\gamma$-доверительного интервала для значения параметра распределения случайной величины}

Дана случайная величина $X$, закон распределения которой известен с точностью до неизвестного параметра $\theta$.

Интервальной оценкой с уровнем доверия $\gamma$ ($\gamma$-доверительной интервальной оценкой) параметра $\theta$ называют пару статистик $\underline{\theta}(\vec X), \overline{\theta}(\vec X)$ таких, что

\begin{equation*}
	P\{\underline{\theta}(\vec X)<\theta<\overline{\theta}(\vec X)\}=\gamma
\end{equation*}

Поскольку границы интервала являются случайными величинами, то для различных реализаций случайной выборки $\vec X$ статистики $\underline{\theta}(\vec X), \overline{\theta}(\vec X)$ могут принимать различные значения.

Доверительным интервалом с уровнем доверия $\gamma$ ($\gamma$-доверительным интервалом) называют интервал $(\underline{\theta}(\vec x), \overline{\theta}(\vec x))$, отвечающий выборочным значениям статистик $\underline{\theta}(\vec X), \overline{\theta}(\vec X)$.

\section{Формулы для вычисления границ \\ $\gamma$-доверительного интервала для математического ожидания и дисперсии нормальной случайной величины}

Формулы для вычисления границ $\gamma$-доверительного интервала для математического ожидания:

\begin{equation}
\underline\mu(\vec X_n)=\overline X + \frac{S(\vec X)t^{St(n-1)}_{\frac{1-\gamma}{2}}}{\sqrt{n}}
\end{equation}

\begin{equation}
\overline\mu(\vec X_n)=\overline X + \frac{S(\vec X)t^{St(n-1)}_{\frac{1+\gamma}{2}}}{\sqrt{n}}
\end{equation}

$\overline X$ -- выборочное среднее;

$S(\vec X) = \sqrtsign{S^2(\vec X)}$ -- квадратный корень из исправленной выборочной дисперсии;

$n$ -- объем выборки;

$\gamma$ -- уровень доверия;

$t^{St(n-1)}_{\alpha}$ -- квантиль уровня $\alpha$ распределения Стьюдента с $n - 1$ степенями свободы.

Формулы для вычисления границ $\gamma$-доверительного интервала для дисперсии:

\begin{equation}
\underline\sigma^2(\vec X_n)= \frac{(n-1)S^2(\vec X)}{t^{\chi^2(n-1)}_{\frac{1+\gamma}{2}}}
\end{equation}

\begin{equation}
\overline\sigma^2(\vec X_n)= \frac{(n-1)S^2(\vec X)}{t^{\chi^2(n-1)}_{\frac{1-\gamma}{2}}}
\end{equation}

$S^2(\vec X)$ -- исправленная выборочная дисперсия;

$n$ -- объем выборки;

$\gamma$ -- уровень доверия;

$t^{\chi^2(n-1)}_{\alpha}$ -- квантиль уровня $\alpha$ распределения $\chi^2(n-1)$ с $n - 1$ степенями свободы.


\chapter*{Результаты работы программы}

\section{Код программы}

\listingfile{main.m}{main}{Matlab}{Файл \texttt{main.m}. }{linerange={6-41}}

\clearpage

\listingfile{main.m}{main}{Matlab}{Файл \texttt{main.m}. }{linerange={43-82},firstnumber=36}

\section{Результаты расчётов}

\begin{align*}
    \hat\mu(\vec x_n) &= -10,132\\
    (\underline\mu(\vec x_n); \overline\mu(\vec x_n)) &= (-10,2709;  -9,9926)\\
\end{align*}

\begin{align*}
    S^2(\vec x_n) &= 0,846\\
    (\underline{S^2}(\vec x_n); \overline{S^2}(\vec x_n)) &= (0,6921; 1,0619)\\
\end{align*}

\imgw{m}{h!}{\textwidth}{Прямая $y(n) = \hat\mu(\vec x_N)$, а также графики функций $y(n) = \underline\mu(\vec x_n)$, $y(n) = \overline\mu(\vec x_n)$, $y(n) = \hat\mu(\vec x_n)$ как функций объема $n$ выборки, где $n$ изменяется от 1 до $N$}

\imgw{s2}{h!}{\textwidth}{Прямая $z(n) = \hat S^2(\vec x_N)$, а также графики функций $z(n) = \underline S^2(\vec x_n)$, $z(n) = \overline S^2(\vec x_n)$, $z(n) = \hat S^2(\vec x_n)$ как функций объема $n$ выборки, где $n$ изменяется от 1 до $N$}
